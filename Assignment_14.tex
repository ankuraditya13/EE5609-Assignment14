\documentclass[journal,12pt,twocolumn]{IEEEtran}

\usepackage{setspace}
\usepackage{gensymb}

\singlespacing


\usepackage[cmex10]{amsmath}

\usepackage{amsthm}

\usepackage{mathrsfs}
\usepackage{txfonts}
\usepackage{stfloats}
\usepackage{bm}
\usepackage{cite}
\usepackage{cases}
\usepackage{subfig}

\usepackage{longtable}
\usepackage{multirow}

\usepackage{enumitem}
\usepackage{mathtools}
\usepackage{steinmetz}
\usepackage{tikz}
\usepackage{circuitikz}
\usepackage{verbatim}
\usepackage{tfrupee}
\usepackage[breaklinks=true]{hyperref}
\usepackage{graphicx}
\usepackage{tkz-euclide}

\usetikzlibrary{calc,math}
\usepackage{listings}
    \usepackage{color}                                            %%
    \usepackage{array}                                            %%
    \usepackage{longtable}                                        %%
    \usepackage{calc}                                             %%
    \usepackage{multirow}                                         %%
    \usepackage{hhline}                                           %%
    \usepackage{ifthen}                                           %%
    \usepackage{lscape}     
\usepackage{multicol}
\usepackage{chngcntr}

\DeclareMathOperator*{\Res}{Res}

\renewcommand\thesection{\arabic{section}}
\renewcommand\thesubsection{\thesection.\arabic{subsection}}
\renewcommand\thesubsubsection{\thesubsection.\arabic{subsubsection}}

\renewcommand\thesectiondis{\arabic{section}}
\renewcommand\thesubsectiondis{\thesectiondis.\arabic{subsection}}
\renewcommand\thesubsubsectiondis{\thesubsectiondis.\arabic{subsubsection}}


\hyphenation{op-tical net-works semi-conduc-tor}
\def\inputGnumericTable{}                                 %%

\lstset{
%language=C,
frame=single, 
breaklines=true,
columns=fullflexible
}
\begin{document}


\newtheorem{theorem}{Theorem}[section]
\newtheorem{problem}{Problem}
\newtheorem{proposition}{Proposition}[section]
\newtheorem{lemma}{Lemma}[section]
\newtheorem{corollary}[theorem]{Corollary}
\newtheorem{example}{Example}[section]
\newtheorem{definition}[problem]{Definition}

\newcommand{\BEQA}{\begin{eqnarray}}
\newcommand{\EEQA}{\end{eqnarray}}
\newcommand{\define}{\stackrel{\triangle}{=}}
\bibliographystyle{IEEEtran}
\providecommand{\mbf}{\mathbf}
\providecommand{\pr}[1]{\ensuremath{\Pr\left(#1\right)}}
\providecommand{\qfunc}[1]{\ensuremath{Q\left(#1\right)}}
\providecommand{\sbrak}[1]{\ensuremath{{}\left[#1\right]}}
\providecommand{\lsbrak}[1]{\ensuremath{{}\left[#1\right.}}
\providecommand{\rsbrak}[1]{\ensuremath{{}\left.#1\right]}}
\providecommand{\brak}[1]{\ensuremath{\left(#1\right)}}
\providecommand{\lbrak}[1]{\ensuremath{\left(#1\right.}}
\providecommand{\rbrak}[1]{\ensuremath{\left.#1\right)}}
\providecommand{\cbrak}[1]{\ensuremath{\left\{#1\right\}}}
\providecommand{\lcbrak}[1]{\ensuremath{\left\{#1\right.}}
\providecommand{\rcbrak}[1]{\ensuremath{\left.#1\right\}}}
\theoremstyle{remark}
\newtheorem{rem}{Remark}
\newcommand{\sgn}{\mathop{\mathrm{sgn}}}
\providecommand{\abs}[1]{\left\vert#1\right\vert}
\providecommand{\res}[1]{\Res\displaylimits_{#1}} 
\providecommand{\norm}[1]{\left\lVert#1\right\rVert}
%\providecommand{\norm}[1]{\lVert#1\rVert}
\providecommand{\mtx}[1]{\mathbf{#1}}
\providecommand{\mean}[1]{E\left[ #1 \right]}
\providecommand{\fourier}{\overset{\mathcal{F}}{ \rightleftharpoons}}
%\providecommand{\hilbert}{\overset{\mathcal{H}}{ \rightleftharpoons}}
\providecommand{\system}{\overset{\mathcal{H}}{ \longleftrightarrow}}
	%\newcommand{\solution}[2]{\textbf{Solution:}{#1}}
\newcommand{\solution}{\noindent \textbf{Solution: }}
\newcommand{\cosec}{\,\text{cosec}\,}
\providecommand{\dec}[2]{\ensuremath{\overset{#1}{\underset{#2}{\gtrless}}}}
\newcommand{\myvec}[1]{\ensuremath{\begin{pmatrix}#1\end{pmatrix}}}
\newcommand{\mydet}[1]{\ensuremath{\begin{vmatrix}#1\end{vmatrix}}}
\numberwithin{equation}{subsection}
\makeatletter
\@addtoreset{figure}{problem}
\makeatother
\let\StandardTheFigure\thefigure
\let\vec\mathbf
\renewcommand{\thefigure}{\theproblem}
\def\putbox#1#2#3{\makebox[0in][l]{\makebox[#1][l]{}\raisebox{\baselineskip}[0in][0in]{\raisebox{#2}[0in][0in]{#3}}}}
     \def\rightbox#1{\makebox[0in][r]{#1}}
     \def\centbox#1{\makebox[0in]{#1}}
     \def\topbox#1{\raisebox{-\baselineskip}[0in][0in]{#1}}
     \def\midbox#1{\raisebox{-0.5\baselineskip}[0in][0in]{#1}}
\vspace{3cm}
\title{Assignment-14}
\author{Ankur Aditya - EE20RESCH11010}
\maketitle
\newpage
\bigskip
\renewcommand{\thefigure}{\theenumi}
\renewcommand{\thetable}{\theenumi}

\begin{abstract}
This document contains the problem related to Linear Transformations (Hoffman:- Page-106,Q-9) 
\end{abstract}
Download the latex-file from 
\begin{lstlisting}
https://github.com/ankuraditya13/EE5609-Assignment14
\end{lstlisting}

\section{Problem}
Let $\vec{V}$ be the vector space of all $2\times 2$ matrices over the field of real numbers, and let
\begin{align}
\vec{B}= \myvec{2&-2\\-1&1}
\end{align}  
Let $\vec{W}$ be the subspace of $\vec{V}$ consisting of all $\vec{A}$ such that $\vec{AB}$=0. Let f be a linear functional on $\vec{V}$ which is an annihilator of $\vec{W}$. Suppose that f($\vec{I}$) = 0 and f($\vec{C}$) = 3, where $\vec{I}$ is the $2\times 2$ identity matrix and,
\begin{align}
\vec{C} = \myvec{0&0\\0&1}
\end{align} 
Find f($\vec{B}$)?
\section{solution}
The general Linear functional f on $\vec{V}$ is of the form,
\begin{align}
f(\vec{A}) = aA_{11}+bA_{12}+cA_{21}+dA_{22}
\label{1}
\end{align}
for a,b,c,d $\in\vec{R}$
Let $\vec{A} \in \vec{W}$ be,
\begin{align}
\vec{A} = \myvec{p&q\\q&s}\\
\because \vec{AB} = 0\\
\implies \myvec{p&q\\q&s}\myvec{2&-2\\-1&1} = 0\\
\implies \myvec{2p-q&-2p+q\\2q-s&-2q+s} = 0
\end{align}
$\therefore$ q=2p and s=2q.
Hence $\vec{W}$ consists of all matrices of the form 
\begin{align}
\myvec{p&2p\\q&2q}
\end{align}
Now V is an annihilator of W. Hence, f$\in\vec{W}^0$
\begin{align}
\implies f\brak{\myvec{p&2p\\q&2q}} = 0 \forall p,q \in \vec{R}
\end{align}
from equation \eqref{1}
\begin{align}
\implies ap+2bp+cq+2dq = 0 \forall p,q \in \vec{R}\\
\implies (a+2b)p + (c+2d)q = 0, \forall p,q \in \vec{R}
\end{align}
Hence, b=$\frac{-1}{2}$a and d=$\frac{-1}{2}$c. Hence general $f\in\vec{W}^0$ is of the form,
\begin{align}
f(\vec{A}) = aA_{11}-\frac{1}{2}a A_{12}+cA_{21}-\frac{1}{2}cA_{22}
\label{2}
\end{align}
Now, f($\vec{C}$) = 3 $\implies$ d =3 $\implies$ c=-6. Alos given that, f($\vec{I}$) = 0 $\implies$ a-$\frac{1}{2}$c =0 $\implies$ a=-3. Substituting the above parameters in equation \eqref{2} we get,
\begin{align}
\therefore f(\vec{A}) = -3A_{11}+\frac{3}{2} A_{12}-6A_{21}+3A_{22} \\
\mbox{Now, } f(\vec{B}) = f\brak{\myvec{2&-2\\-1&1}} 
\end{align}
\begin{align}
\implies f(\vec{B}) = -3(2)+\frac{3}{2}(-2)-6(-1)+3(1) = 0 
\end{align}
\end{document}